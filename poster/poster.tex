%Poster do trabalho de conclusao de curso 

\documentclass[final]{beamer}
\mode<presentation>{\usetheme{azul}}
\usepackage{graphicx}
\usepackage{epstopdf}
\usepackage{subfigure}

\usepackage[brazil]{babel}
\usepackage[utf8]{inputenc}
\usepackage{ragged2e} 

\usepackage[T1]{fontenc}
\usepackage[justification=centering]{caption}


\usepackage{amsmath,amsthm, amssymb, latexsym}
\usepackage[orientation=portrait,size=a2,scale=1.4]{beamerposter}
\usepackage[ruled]{algorithm2e}

\usepackage{snapshot} % will write a .dep file with all dependencies, allows for easy bundling

\DeclareMathSizes{17.42}{15}{14}{10}  % Math text size

%%%%%%%%%%%%%%%%%%%%%%%%%%%%%%%%
%%  MACROS %%%%%%%%%%%%%%%%%%%%%
\usepackage{xspace}
\newcommand{\pixel}{\emph{pixel}\xspace}
\newcommand{\pixels}{\emph{pixels}\xspace}
\newcommand{\voxel}{\emph{voxel}\xspace}
\newcommand{\voxels}{\emph{voxels}\xspace}


\listfiles
%%%%%%%%%%%%%%%%%%%%%%%%%%%%%%%%%%%%%%%%%%%%%%%%%%%%%%%%%%%%%%%%%%%%%%%%%%%%%%%%%%%%%%
\title{\huge yay}

\author{eu<eu@eu.eu>, Orientador: ele<ele@ele.ele>}
\institute[Universidade de São Paulo] % (optional, but mostly needed)
{
  Instituto de Matemática e Estatística, Universidade de São Paulo - Trabalho
  de Conclusão de Curso
}

\date[Novembro 2015]{Novembro, 2015}
%%%%%%%%%%%%%%%%%%%%%%%%%%%%%%%%%%%%%%%%%%%%%%%%%%%%%%%%%%%%%%%%%%%%%%%%%%%%%%%%%%%%%%
\newlength{\columnheight}
\setlength{\columnheight}{65cm}
%%%%%%%%%%%%%%%%%%%%%%%%%%%%%%%%%%%%%%%%%%%%%%%%%%%%%%%%%%%%%%%%%%%%%%%%%%%%%%%%%%%%%%
\begin{document}
\begin{frame}
  \begin{columns}
    % ---------------------------------------------------------%
    % Set up a column 
    \begin{column}{.5\textwidth}
      \begin{beamercolorbox}[center,wd=\textwidth]{postercolumn}
        \begin{minipage}[T]{.95\textwidth} % tweaks the width, makes a new \textwidth
          \parbox[t][\columnheight]{\textwidth}{ % must be some better way to set the the height, width and textwidth simultaneously
            % Since all columns are the same length, it is all nice and tidy.  You have to get the height empirically
            % ---------------------------------------------------------%
            % fill each column with content            
            
            \vspace*{0.8cm}
            
            \begin{block}{Introdução}
            \justifying
                Something something autotuning. Something something something.
                
                \vspace*{0.15cm}
                
                Something something OpenTuner. Something something.
                
                \vspace*{0.15cm}

                Something something OpenTTD. Something Something

%                \textbf{AAAAAAAAAAAAAAAAAAA}
                
%                \vspace*{0.2cm}
            \end{block}
            
            \vspace*{0.2cm}

            \begin{block}{Objetivos}
              \justifying
              \begin{itemize}
                \item Something something viabilidade 
                
                \vspace*{0.4cm}
                
                \item Implementar pelo menos 1 prova de conceito.
              \end{itemize}
              \vspace*{0.2cm} 
            \end{block}
            
            \vspace*{0.2cm}
            
            \begin{block}{Rede peer-to-peer}
                \center 
%                \begin{figure}[h]
%                  \fbox{\includegraphics[width=0.45\textwidth]{p2p-network.png}}
%                  \fbox{\includegraphics[width=0.45\textwidth]{server-based-network.png}}
%                  \caption{Comparação de rede Peer-to-peer (Esquerda) com Cliente-Servidor (Direita). 
%                    Fonte: Baseado em figura de Wikimedia Commons}
%                \end{figure}
                                
                \vspace*{0.2cm}
            \end{block}
            
            \vspace*{0.2cm}

            \begin{block}{Protocolo de Rede}
            \justifying
                Quando mensagens são trocadas através de uma rede de computadores seguindo um sistema de regras digitais,
                tal sistema é chamado de um protocolo de rede. Sistemas de comunicação usam formatos bem-definidos para
                trocar mensagens. 
                
                \vspace*{0.4cm} 
                
%                \begin{figure}[h]
%                  \fbox{\includegraphics[width=0.4\textwidth]{passive_ftp_connection.png}}
%                  \caption{Exemplo de fluxo de mensagens no protocolo FTP.
%                    Fonte: Wikimedia Commons}
%                \end{figure}
                
                \vspace*{0.2cm} 
                
                Cada mensagem tem um significado exato intencionado a provocar uma resposta particular
                do receptor. Portanto, o protocolo define a sintaxe, semânticas, e sincronização da comunicação. Protocolos
                de comunicação devem ser acordados por todas as partes envolvidas.
                               
                \vspace*{0.2cm} 
            \end{block}
          }
        \end{minipage}
      \end{beamercolorbox}
    \end{column}
    % ---------------------------------------------------------%
    % end the column

    % ---------------------------------------------------------%
    % Set up a column 
    \begin{column}{.5\textwidth}
      \begin{beamercolorbox}[center,wd=\textwidth]{postercolumn}
        \begin{minipage}[T]{.95\textwidth} % tweaks the width, makes a new \textwidth
          \parbox[t][\columnheight]{\textwidth}{ % must be some better way to set the the height, width and textwidth simultaneously
            % Since all columns are the same length, it is all nice and tidy.  You have to get the height empirically
            % ---------------------------------------------------------%
            % fill each column with content
            
            \vspace*{0.8cm}
            
            \begin{block}{Etherclan}
                NOES
                \begin{itemize}
                  \item Não depende de um servidor centralizado.
                  \item Pode ser utilizada por mais de um jogo simultaneamente.
                  \item Capaz de buscar por membros de um jogo específico.
                \end{itemize}
                
                \vspace*{0.5cm}
                Para que os nós consigam conversar entre si, foi necessário definir um protocolo. 
                Nele, foram existem as seguintes mensagems:
                \begin{itemize}
                  \item ANNOUNCE\_SELF -- Anuncia a própria existência para o receptor.
                  \item REQUEST\_NODE\_LIST -- Requisita do receptor uma lista de nós da rede.
                  \item KEEP\_ALIVE -- Avisa que vai enviar mais mensagens nessa mesma conexão.
                  \item SERVICE -- Comando para ser tratado usando alguma extensão definida por terceiros.
                \end{itemize}
                
                \vspace*{0.2cm}
                
                E a seguinte resposta possível:
                \begin{itemize}
                  \item NODE\_INFO -- Informações sobre um nó da rede.
                \end{itemize}
                
                \vspace*{0.2cm} 
            \end{block}

            \vspace*{0.2cm} 
            
            \begin{block}{Resultados: Chat}
              \justifying 
                Como prova de conceito, foi implementado um programa de conversação que utiliza a rede para enviar e receber
                mensagens de outros nós. As mensagens são enviadas para todos e sem nome, com o único identificador sendo o IP
                de origem.
                
%                \begin{figure}[htp]
%                  \centering
%                   \fbox{\includegraphics[width=0.7\textwidth]{chat.png}}
%                  \caption{Chat em funcionamento}
%                \end{figure}
%                \begin{figure}[htp]
%                  \centering
%                   \fbox{\includegraphics[width=0.95\textwidth]{chat-log.png}}
%                  \caption{Log das mensagens enviadas e recebidas durante a descoberta de nós}
%                \end{figure}
                
                \vspace*{0.4cm}
                
                Resultados obtidos com esse exemplo incluem:
                \begin{itemize}
                  \item Funciona! É possível conseguir encontrar a rede inteira através de alguns nós.
                  \item Problemas com NAT. Sem adotar pelo menos UPnP, torna-se impraticável o uso para encontrar nós atravéz da Internet.
                  \item É necessário ter boas escolhas em quando iniciar novas buscas por nós, assim como por quanto tempo executar a busca.
                  \item Nós não avisam quando eles ficam offline, então algum critério para decidir se um nó deixou de existir é importante.
                \end{itemize}
                
                \vspace*{0.2cm}
            \end{block}
            
            \vspace*{0.2cm} 
            \begin{block}{Trabalhos Futuros}
                Durante o desenvolvimento do protocolo e principalmente da prova de conceito, vários pontos se mostraram fundamentais e que serão estudados no futuro.
                
                \begin{itemize}
                  \item Política de quantos outros nós buscar.
                  \item Política de quando que um nó é considerado morto e deve ser ignorado.
                  \item Diminuir a complexidade de para quais nós devo perguntar para algo melhor que O(n).
                  \item Desenvolvimento de uma biblioteca em C++ para facilitar a integração com outros projetos.
                \end{itemize}
                
                \vspace*{0.2cm} 
            \end{block}
            
            %\vspace*{0.2cm} 
            %
            %\begin{block}{Referências}
            %  \small
            %    \begin{itemize}
            %        \item Rüdiger Schollmeier, A Definition of Peer-to-Peer Networking for the Classification of Peer-to-Peer Architectures and Applications, Proceedings of the First International Conference on Peer-to-Peer Computing, IEEE (2002).
            %    \end{itemize}
            %    \vspace*{0.2cm} 
            %\end{block}
            \vfill
          }
        \end{minipage}
      \end{beamercolorbox}
    \end{column}
    % ---------------------------------------------------------%
    % end the column


  \end{columns}
\end{frame}

\end{document}


%%%%%%%%%%%%%%%%%%%%%%%%%%%%%%%%%%%%%%%%%%%%%%%%%%%%%%%%%%%%%%%%%%%%%%%%%%%%%%%%%%%%%%%%%%%%%%%%%%%%
%%% Local Variables: 
%%% mode: latex
%%% TeX-PDF-mode: t
%%% End:
