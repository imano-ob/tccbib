%Poster do trabalho de conclusao de curso 

\documentclass[final]{beamer}
\mode<presentation>{\usetheme{azul}}
\usepackage{graphicx}
\usepackage{epstopdf}
\usepackage{subfigure}

\usepackage[brazil]{babel}
\usepackage[utf8]{inputenc}
\usepackage{ragged2e} 

\usepackage[T1]{fontenc}
\usepackage[justification=centering]{caption}


\usepackage{amsmath,amsthm, amssymb, latexsym}
\usepackage[orientation=portrait,size=a2,scale=1.4]{beamerposter}
\usepackage[ruled]{algorithm2e}

\usepackage{snapshot} % will write a .dep file with all dependencies, allows for easy bundling

\DeclareMathSizes{17.42}{15}{14}{10}  % Math text size

%%%%%%%%%%%%%%%%%%%%%%%%%%%%%%%%
%%  MACROS %%%%%%%%%%%%%%%%%%%%%
\usepackage{xspace}
\newcommand{\pixel}{\emph{pixel}\xspace}
\newcommand{\pixels}{\emph{pixels}\xspace}
\newcommand{\voxel}{\emph{voxel}\xspace}
\newcommand{\voxels}{\emph{voxels}\xspace}


\listfiles
%%%%%%%%%%%%%%%%%%%%%%%%%%%%%%%%%%%%%%%%%%%%%%%%%%%%%%%%%%%%%%%%%%%%%%%%%%%%%%%%%%%%%%
\title{\huge Aplicando o Arcabouço OpenTuner em Jogos Digitais}

\author{Renan Teruo Carneiro, Vitor Cerqueira Santos <renantc@linux.ime.usp.br varyag@linux.ime.usp.br>, Orientador: Alfredo Goldman <gold@ime.usp.br>}
\institute[Universidade de São Paulo] % (optional, but mostly needed)
{
  Instituto de Matemática e Estatística, Universidade de São Paulo - Trabalho
  de Conclusão de Curso
}


\date[Novembro 2015]{Novembro, 2015}
%%%%%%%%%%%%%%%%%%%%%%%%%%%%%%%%%%%%%%%%%%%%%%%%%%%%%%%%%%%%%%%%%%%%%%%%%%%%%%%%%%%%%%
\newlength{\columnheight}
\setlength{\columnheight}{65cm}
%%%%%%%%%%%%%%%%%%%%%%%%%%%%%%%%%%%%%%%%%%%%%%%%%%%%%%%%%%%%%%%%%%%%%%%%%%%%%%%%%%%%%%
\begin{document}
\begin{frame}
  \begin{columns}
    % ---------------------------------------------------------%
    % Set up a column 
    \begin{column}{.5\textwidth}
      \begin{beamercolorbox}[center,wd=\textwidth]{postercolumn}
        \begin{minipage}[T]{.95\textwidth} % tweaks the width, makes a new \textwidth
          \parbox[t][\columnheight]{\textwidth}{ % must be some better way to set the the height, width and textwidth simultaneously
            % Since all columns are the same length, it is all nice and tidy.  You have to get the height empirically
            % ---------------------------------------------------------%
            % fill each column with content            
            
            \vspace*{0.8cm}
            
            \begin{block}{Introdução}
            \justifying
                Neste trabalho são exploradas possíveis aplicações do conceito de Ajuste Fino, ou \textit{autotuning} em jogos digitais. 
                                
                \vspace*{0.15cm}
                
                O processo do \textit{autotuning} consiste em automatizar a otimização de um programa alterando seus parâmetros ou algoritmos. Um sistema que realiza essa ação é comumente chamado de \textit{autotuner}, e pode funcionar de várias maneiras. Ao testar diversas configurações diferentes de parâmetros ou algoritmos, um autotuner escolhe as que obtêm o melhor desempenho para o programa. O desempenho pode ser medido pelo tempo de execução, ou pela maximização da saída do programa, por exemplo.
                
                \vspace*{0.15cm}
                
                Os possíveis usos e aplicações de um \textit{autotuner} no campo de jogos digitais são diversos, desde a criação de ferramentas externas para jogadores que desejam conseguir o máximo de seus pontos em um jogo, até possíveis implementações de meta-ferramentas de análise e desenvolvimento de jogos.
                
                \vspace*{0.15cm}
                
                Neste trabalho foi utilizado o OpenTuner, um arcabouço de ajuste fino para otimização de parâmetros. Analisando os resultados gerados, o \textit{autotuner} explora o espaço de busca, encontrando otimizações para o programa.
                
                \vspace*{0.15cm}
                
                O OpenTTD é um simulador de gerência de uma empresa de transportes, onde é possível construir e controlar rotas de entrega, linhas de trem e outros meios de transporte. O jogo utiliza Inteligências Artificiais que controlam empresas. Neste trabalho, implementamos um \textit{autotuner} para as IAs do OpenTTD, e analisamos o desempenho das IAs em relação aos objetivos do jogo.

%                \textbf{AAAAAAAAAAAAAAAAAAA}
                
%                \vspace*{0.2cm}
            \end{block}
            
            \vspace*{0.2cm}

            \begin{block}{Objetivos}
              \justifying
              \begin{itemize}
                \item Estudar a viabilidade e o desempenho de aplicação de técnicas de \textit{autotuning} em jogos digitais.
                \item Implementar um \textit{autotuner} aplicado às IAs do jogo OpenTTD.
                \item Estudar o desempenho de IAs, em relação aos objetivos do OpenTTD, como o lucro da empresa.
              \end{itemize}
              \vspace*{0.2cm} 
            \end{block}
            
            \vspace*{0.2cm}

            \begin{block}{OpenTuner}
            \justifying
                OpenTuner é um arcabouço para a implementação de sistemas de otimização e autotuning de programas. Utilizando um conjunto de técnicas empíricas de busca, o OpenTuner gera e testa combinações de parâmetros de configuração para um determinado programa, que podem representar escolhas algorítmicas. 
                
                \vspace*{0.2cm}
                
                O OpenTuner realiza a ação de otimização via um conjunto de métodos. O módulo de medição recebe objetivos de diferentes tipos, como por exemplo tempo de execução, e tenta minimizar ou maximizar esses objetivos. Ao final da execução, os resultados observados são devolvidos para o módulo de otimização, que navega o espaço de busca do problema, e chega em novos resultados, que são enviados novamente ao módulo de medição.
                
                \vspace*{0.4cm} 
                
                
                \begin{figure}[h]
                  \fbox{\includegraphics[width=0.6\textwidth]{OpenTuner.png}}
                  \caption{Método de funcionamento do OpenTuner.
                    Fonte: Ansel, Jason, et al, Opentuner: An extensible framework for program autotuning}
                \end{figure}
                
                \vspace*{0.2cm} 
                
                 A ideia básica  de aplicar o OpenTuner a um jogo é que podemos usá-lo para otimizar um conjunto de parâmetros e ajudar o jogador a planejar ou desenvolver seu jogo.
                 
                 \vspace*{0.2cm} 
                 
                 Implementamos um \textit{autotuner} para as IAs do jogo OpenTTD, que otimiza os parâmetros do algoritmo de \textit{pathfinding} de uma IA. As IAs produzidas pelo \textit{autotuner} foram avaliadas segundo o valor da empresa produzida e ganho monetário.
                               
                \vspace*{0.2cm} 
            \end{block}
                        
            \vspace*{0.2cm}
          }
        \end{minipage}
      \end{beamercolorbox}
    \end{column}
    % ---------------------------------------------------------%
    % end the column

    % ---------------------------------------------------------%
    % Set up a column 
    \begin{column}{.5\textwidth}
      \begin{beamercolorbox}[center,wd=\textwidth]{postercolumn}
        \begin{minipage}[T]{.95\textwidth} % tweaks the width, makes a new \textwidth
          \parbox[t][\columnheight]{\textwidth}{ % must be some better way to set the the height, width and textwidth simultaneously
            % Since all columns are the same length, it is all nice and tidy.  You have to get the height empirically
            % ---------------------------------------------------------%
            % fill each column with content
            
            \vspace*{0.8cm}
            
            \begin{block}{OpenTTD}
                \textbf{O Jogo}
                
                  OpenTTD é um simulador de gerência de uma empresa de transportes, onde pode-se comprar, vender e construir veículos e linhas de transporte, como por exemplo trens. O jogo possui um modo para vários jogadores via rede com um servidor, e suporta Inteligências Artificiais.
                
                \vspace*{0.2cm}
                Os testes do \textit{autotuner} executaram e avaliaram o desempenho das IAs no ambiente do jogo.
                \begin{itemize}
                  \item Utilizamos uma IA chamada ChooChoo, especializada no gerenciamento e construção de linhas de trem.
                  \item O \textit{autotuner} tenta otimizar os parâmetros da função de \textit{pathfinding} da IA.
                \end{itemize}  
                         
                \vspace*{0,2cm}
                
                \textbf{Experimentos}
                
                \vspace*{0.2cm}
                A execução dos testes consiste em iniciar um servidor do jogo, inserir uma IA no ambiente do jogo usando os parâmetros dados pelo \textit{autotuner}, e então avaliar o ganho monetário da empresa representada pela IA. Alterando os valores dos custos no algoritmo de \textit{pathfinding}, o \textit{autotuner} navega o espaço de busca do problema procurando pela configuração que maximiza o ganho monetário e o valor da empresa ao final de um período dentro do jogo.
                
                \vspace*{0.5cm} 
            \end{block}

            \vspace*{0.2cm} 
            
            \begin{block}{Resultados}
              \justifying 
                A implementação é composta de três partes:
                \begin{itemize}
                	\item O construtor gera os arquivos a serem alterados para os testes e os passa para o \textit{autotuner}
                	\item O handler organiza e trata a saída do programa, comunicando-se com o \textit{autotuner}
                	\item O \textit{autotuner} executa os testes em um servidor de jogo, carregando os arquivos de IA modificados, e registrando seus resultados.
                \end{itemize}
                
                \vspace*{0.2cm}
                Novos parâmetros são gerados baseando-se nos resultados encontrados até o momento, e as informações necessárias para preparar os novos arquivos de teste são passadas para o construtor.

                \begin{figure}[htp]
                  \centering
                   \fbox{\includegraphics[width=0.6\textwidth]{value-50yrs-best.png}}
                  \caption{Teste com duração de 50 anos dentro do jogo}
                \end{figure}
                
                \vspace*{0.2cm}
                
                Resultados obtidos com esse exemplo incluem:
                \begin{itemize}
                  \item É possível notar uma melhora gradual nos resultados do valor da empresa, depois da realização dos testes.
                  \item Há um impacto significativo no crescimento do valor da empresa, o que é surpreendente dado o escopo do alvo da otimização: os custos do algoritmo de \textit{pathfinding}.
                  \item Outros valores também foram avaliados, como o lucro final e o valor de caixa ao final do semestre, e também foram observadas significativas melhoras.
                  
                  \vspace*{0.2cm}
                  
                  \textbf{Dificuldades}
                  \item Testes em paralelo com o OpenTuner não foram implementados até o presente momento, o que causa uma geração mais lenta de resultados, e uma navegação mais lenta do espaço de busca do algoritmo.
                \end{itemize}
                
                \vspace*{0.2cm}
            \end{block}
            
%            \vspace*{0.2cm} 
%            \begin{block}{Trabalhos Futuros}
%                Durante o desenvolvimento do tuner, 
%                
%                \begin{itemize}
%                  \item 
%                  \item 
%                 
%                \end{itemize}
%                
%                \vspace*{0.2cm} 
%                
%            \end{block}
            
            \vspace*{0.2cm} 
            
            \begin{block}{Referências}
              \small
                \begin{itemize}
                    \item Ansel, Jason, et al. "Opentuner: An extensible framework for program autotuning." Proceedings of the 23rd international conference on Parallel architectures and compilation. ACM, 2014.
                \end{itemize}
                \vspace*{0.2cm} 
            \end{block}
            \vfill
          }
        \end{minipage}
      \end{beamercolorbox}
    \end{column}
    % ---------------------------------------------------------%
    % end the column


  \end{columns}
\end{frame}

\end{document}


%%%%%%%%%%%%%%%%%%%%%%%%%%%%%%%%%%%%%%%%%%%%%%%%%%%%%%%%%%%%%%%%%%%%%%%%%%%%%%%%%%%%%%%%%%%%%%%%%%%%
%%% Local Variables: 
%%% mode: latex
%%% TeX-PDF-mode: t
%%% End:
