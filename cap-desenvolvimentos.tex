%% ------------------------------------------------------------------------- %%
%Parte Objetiva

%\chapter{Desenvolvimentos}
%\label{cap:desenvolvimentos}
\chapter{Otimiza��o de Par�metros}
%-mostrar um pouco da base te�rica pesquisada, com exemplos relevantes dos artigos citados
%-mencionar trabalhos relacionados, bases te�ricas de AutoTuning
\section{O OpenTuner}
%-explicar o que �, e o que esperamos dele
%-dar exemplos mais detalhados
%-incluir aqui nosso exemplo das �rvores de habilidades

\chapter{OpenTTD}
%-explicar os b�sicos sobre o jogo
%-explicar os nosso objetivos em aplicar o OpenTuner sobre ele
%-explicar como exatamente integramos os dois

\chapter{Experimentos}
%-coisas que aconteceram
Embora neste exemplo tenhamos apenas um cap�tulo,  entre a introdu��o
e a conclus�o de uma monografia podemos ter uma sequ�ncia de cap�tulos
descrevendo o trabalho e os resultados. Estes podem descrever
fundamentos, trabalhos relacionados, m�todo/modelo/algoritmo proposto,
experimentos realizados, resulatdos obtidos.

Cada cap�tulo pode ser organizado em se��es, que por sua vez pode
conter subse��es. 

%Um exemplo de figura est� na figura~\ref{fig:graph}.
%\begin{figure}[htb]
%\includegraphics[width=5cm]{figuras/graph}
%\caption{\label{fig:graph}Exemplo de uma figura.}
%\end{figure}
